\Section
	{Educación}
	{Educación}
	{PDF:Educación}

%%%%%%%%%%%%%%%%%%%%
%%% Preparatória
%%%%%%%%%%%%%%%%%%%%

%\Entry
\href{https://preparatoria7.uanl.mx}
{\textbf{Preparatoria:}}
\hfill
\DatestampYMD{2013}{08}{15} --
\DatestampYMD{2015}{08}{15}
\Gap
Preparatoria № 7 Puentes, Av. las Puentes s/n, Las Puentes 1er Sector, 66460 San Nicolás de los Garza, N.L.

%\BigGap


%%%%%%%%%%%%%%%%%%%%
%%% Facultad
%%%%%%%%%%%%%%%%%%%%

\Entry
	\href{https://www.uanl.mx}
	{\textbf{Universidad:}}
	\hfill
	\Gap
	Universidad Autónoma de Nuevo León, Pedro de Alba s/n, Niños Héroes, Ciudad Universitaria, San Nicolás de los Garza, N.L.

\Gap

\BulletItem
	\textbf{\textit{Facultad de Ciencias Químicas:}}
	\href{http://www.fcq.uanl.mx/}
	{Licenciatura en Química Industrial}
	\hfill
	%\DatestampYMD{2015}{08}{15} --
	\DatestampY{2022}
	\begin{Detail}
		\SubBulletItem
		Tesis:
		\href{https://www.icloud.com/iclouddrive/0cfgD6maOU4-X1y5lNikwk4eQ#PEAG-Tesis_final}
		{Predicción \textit{in-silico} de los mecanismos de la glicólisis de polietilentereftalato con sales de zinc como catalizadores: un estudio en DFT ωB97X-D4/def2-TZVP}
		\SubSubBulletItem
		Asesor:
		Dra. Isabel del Carmen Sáenz Tavera
		\SubSubBulletItem
		Co-Asesor:
		Dr. Víctor Manuel Rosas García
		\SubSubBulletItem
		Enfoque:
		Química computacional, DFT, depolimerización, catálisis.

    \Gap

		\SubBulletItem
		\textbf{Examen de egreso (CENEVAL):}\\
		\textit{\textbf{Promedio:}} Sobresaliente\\
		\textit{\textbf{Identificación IES:}} 212655
		\SubSubBulletItem Sección transversal del lenguaje y comunicación: Sobresaliente;
		\SubSubBulletItem EGEL-QUIM: Sobresaliente.
	\end{Detail}

