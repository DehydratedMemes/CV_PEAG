\newpage
\Section
{Experiencia labroal}
{Experiencia labroal}
{PDF:ExperienciaLabroal}

% \Entry
% {\textbf{Sociedad Estudiantil: SELQI}},
% \newline
% Facultad de Ciencias Químicas, UANL

% \Gap
% \BulletItem
% Miembro
% \hfill
% \DatestampY{2019} -- \DatestampY{2020}

\Entry
{\textbf{Red de Fríos S.A. de C.V.}},
\newline
I. Comonfort, Bella Vista, 64410 Monterrey, N.L.

\Gap
\BulletItem
Jefe de aseguramiento de calidad
\hfill
\DatestampYYYYMM{2022-05} -- \textit{Actualmente}
\newline
\BulletItem \textbf{Actividades:}
Las actividades cotidianas de mi departamento involucran: \compac elaboración y mantenimiento de información documentada, implementación y verificación de instrucciones de trabjo, implementación de programas (HACCP, VACCP, Food Fraud, Food Defense), generación y mantenimiento de formularios y sus versiones.

\newline
\BulletItem \textbf{Implementaciones al sistema de gestión de calidad:}
Hasta la fecha me he encargado de llevar al día los registros generados a partir de procedimientos establesidos en el SGC. A su vez he implementado al SGC lo siguiente:
\begin{compactitem}[$\textendash$]
    \item \textbf{Control de versiones:} Mediante el uso de GIT, desarrollé un control de versiones para los formularios existentes e información documentada en general; 
    \item \textbf{digitalización de registros:} previamente los registros generados se mantenían exclusivamente en físico, con el propósito de prevenir perdida de información, digitalicé a partir de mi entrada todos los registros generados;
    \item \textbf{digitalización de registros:} previamente, la información documentada era difícil de consultar debido a que se encontraba en un formato inadecuado (*.docx) y presentaba referencias a articulos y a procedimientos obsoletos. Impulsé a que se cambiaran los procedimientos a Markdown para su eficaz consulta y a (Xe)LaTeX para la genración de la información documentada física. Desde entonces, la actualización de procedimientos y de folios se ha vuelto más sencilla;
    \item \textbf{bases de datos:} implementé una serie de bases de datos con el propósito de controlar más eficazmente la información obtenida de los registros;
    \item \textbf{análisis estadisticos:} implementé el uso de pruebas estadisticas frecuentistas para verificar la variabilidad entre termómetros y entre mediciones a diferentes horas del día;
    \item \textbf{calibración de equipos de medición de temperatura:} verifico y calibro constantemente los termómetros con los que trabajamos (Termómetros IR, de bástaso, de alcohol y termopares).
\end{compactitem}