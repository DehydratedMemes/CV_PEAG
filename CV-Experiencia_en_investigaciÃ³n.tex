\Section
{Experiencia en investigación}
{Experiencia en investigación}
{PDF:Experiencia en investigación}

\Entry
\href{http://www.fcq.uanl.mx/oferta-educativa/posgrado/}
{\textbf{División de Estudios de Posgrado, Facultad de Ciencias Químicas}},
UANL

\Gap
\BulletItem
Verano de investigación PROVERICYT
\hfill
\DatestampYMD{2017}{04}{1} --
\DatestampYMD{2017}{11}{1}
\begin{Detail}
	\SubBulletItem
	Proyecto:
	Síntesis de nanopartículas de \ch{FeS2} y \ch{NiO} por sonoquímica asistida con líquido iónico y evaluación de su desempeño como cátodo y ánodo en baterías de ión litio, respectivamente.
	\SubBulletItem
	Supervisor:
	Dra.~Salomé Maribel de la Parra Arciniega\\
	Colaboradores: Ariel Gonzalo Godoy Palominos; Mario Alberto Solano Esquer; Dr. Ignacio González Martínez; Dra. Rosa Martha Jiménez Barrera.
	\SubBulletItem
	Enfoque:
	Nanopartículas, sonoquímica, baterías, liquidos iónicos.
\end{Detail}

\Gap
\BulletItem
Estancia de investigación avanzada
\hfill
\DatestampYMD{2019}{08}{1} -- Presente
\begin{Detail}
	\SubBulletItem
	Proyecto:
	Predicción \textit{in-silico} de los mecanismos de la glicólisis de polietilentereftalato con sales de zinc como catalizadores: un estudio en DFT ωB97X-D4/def2-TZVP.
	\SubBulletItem
	Supervisor:
	Dra. Isabel del Carmen Sáenz Tavera\\
	Colaboradores: Dr. Víctor Manuel Rosas García.
	\SubBulletItem
	Enfoque:
	Química computacional, DFT, depolimerización, catálisis.
\end{Detail}
