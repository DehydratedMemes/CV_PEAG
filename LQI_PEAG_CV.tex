% !TEX root = LQI_PEAG_CV.tex
% !TEX TS-program = xelatex
% !TEX encoding = UTF-8 Unicode
% -*- coding: UTF-8; -*-
% vim: set fenc=utf-8

%%%%%%%%%%%%%%%%%%%%%%%%%%%%%%%%%%%%%%%%%%%%%%%%%%%%%%%%%%%%%%%%%
%% CV.tex
%% <https://github.com/zachscrivena/simple-resume-cv>
%% This is free and unencumbered software released into the
%% public domain; see <http://unlicense.org> for details.
%%%%%%%%%%%%%%%%%%%%%%%%%%%%%%%%%%%%%%%%%%%%%%%%%%%%%%%%%%%%%%%%%

% See "README.md" for instructions on compiling this document.

\documentclass[letterpaper,yyyyMMdd,nonstopmode]{simpleresumecv}
% Class options:
% a4paper, letterpaper, nonstopmode, draftmode
% MMMyyyy, ddMMMyyyy, MMMMyyyy, ddMMMMyyyy, yyyyMMdd, yyyyMM, yyyy

%%%%%%%%%%%%%%%%%%%%%%%%%%%%%%%%%%%%%%%%%%%%%%%%%%%%%%%%%%%%%%%%%
%% PREAMBLE.
%%%%%%%%%%%%%%%%%%%%%%%%%%%%%%%%%%%%%%%%%%%%%%%%%%%%%%%%%%%%%%%%%

\usepackage{chemmacros}
\usepackage{polyglossia}
\setmainlanguage{spanish}
\usepackage{paralist}
\usepackage{mathspec}
	\setmainfont{EB Garamond}
	\setmathfont{Garamond-Math}

% CV Info (to be customized).
\newcommand{\CVAuthor}{Pablo Ernesto Alanís González}
\newcommand{\CVTitle}{Aplicación para \textit{Quality Control Chemical,} Symrise}
%\newcommand{\CVNote}{CV compiled on {\today} for Acme Corporation}
%\newcommand{\CVWebpage}{http://www.example.com/~johndoe}
%\usepackage[dvipsnames]{xcolor}
% PDF settings and properties.

%\usepackage[svgnames]{color}

\hypersetup{
pdftitle={\CVTitle},
pdfauthor={\CVAuthor},
%pdfsubject={},
pdfcreator={XeLaTeX},
pdfproducer={Pablo Alanis},
pdfkeywords={CV,2022},
unicode=true,
bookmarks=true,
bookmarksopen=true,
pdfstartview=FitH,
pdfpagelayout=OneColumn,
pdfpagemode=UseOutlines,
breaklinks,
urlbordercolor=green}

% Shorthand.
\newcommand{\Code}[1]{\mbox{\textbf{#1}}}
\newcommand{\CodeCommand}[1]{\mbox{\textbf{\textbackslash{#1}}}}

%%%%%%%%%%%%%%%%%%%%%%%%%%%%%%%%%%%%%%%%%%%%%%%%%%%%%%%%%%%%%%%%%
%% ACTUAL DOCUMENT.
%%%%%%%%%%%%%%%%%%%%%%%%%%%%%%%%%%%%%%%%%%%%%%%%%%%%%%%%%%%%%%%%%

\begin{document}

%%%%%%%%%%%%%%%
% TITLE BLOCK %
%%%%%%%%%%%%%%%

\Title{\CVAuthor}
\SubTitle{\CVTitle}
\begin{SubTitle}
	\href{https://goo.gl/maps/thfLEMdRCJN7BG1a7}
	{Australia 13, Valle del Nogalar, 66480 San Nicolas de Los Garza, NL, México}
	\par
	\href{mailto:pabloalanis1998@gmail.com}
	{pabloalanis1998@gmail.com}
	\,\SubBulletSymbol\,
	(+52)\,81 1177--3600
	\,\SubBulletSymbol\,
	\href{https://www.linkedin.com/in/pabloalanis/}{LinkedIn: in/pabloalanis}
\end{SubTitle}

\begin{Body}

	%%%%%%%%%%%%%%%%%%%%%%%%%%%%%
	%%%   Resumen profesional
	%%%%%%%%%%%%%%%%%%%%%%%%%%%%%
	\Section
	{Resumen profesional}
	{Resumen profesiona}
	{PDF:Resumen profesional}

	\Entry
	%Soy una persona muy proactiva y aprendo rápidamente; mi tesis, por ejemplo, fue en el área de química computaciónal (DFT), un tema que no se aborda en la carrera. Escogí esa disciplina puesto que también me gusta la programación y soy bueno utilizando la tecnología.
	%Me considero una persona comunicativa y me beneficio de trabajar en equipos, sobretodo si son equipos con diferentes disciplinas para resolver problemas. 
	Soy una persona altamente proactiva, aprendo con facilidad y me gusta trabajar en equipos. En mi actual trabajo soy jefe de aseguramiento de calidad en un almacén frío que se encarga de almacenar y procurar la inocuidad de productos alimenticios de empresas internacionalmente reconocidas.
	
	\hfill


%%%%%%%%%%%%%%%%%%%%%%%%%%%%%%%%%%
%%% Formación académica
%%%%%%%%%%%%%%%%%%%%%%%%%%%%%%%%%%

\Section
	{Educación}
	{Educación}
	{PDF:Educación}

%%%%%%%%%%%%%%%%%%%%
%%% Preparatória
%%%%%%%%%%%%%%%%%%%%

%\Entry
\href{https://preparatoria7.uanl.mx}
{\textbf{Preparatoria:}}
\hfill
\DatestampYMD{2013}{08}{15} --
\DatestampYMD{2015}{08}{15}
\Gap
Preparatoria № 7 Puentes, Av. las Puentes s/n, Las Puentes 1er Sector, 66460 San Nicolás de los Garza, N.L.

%\BigGap


%%%%%%%%%%%%%%%%%%%%
%%% Facultad
%%%%%%%%%%%%%%%%%%%%

\Entry
	\href{https://www.uanl.mx}
	{\textbf{Universidad:}}
	\hfill
	\Gap
	Universidad Autónoma de Nuevo León, Pedro de Alba s/n, Niños Héroes, Ciudad Universitaria, San Nicolás de los Garza, N.L.

\Gap

\BulletItem
	\textbf{\textit{Facultad de Ciencias Químicas:}}
	\href{http://www.fcq.uanl.mx/}
	{Licenciatura en Química Industrial}
	\hfill
	%\DatestampYMD{2015}{08}{15} --
	\DatestampY{2022}
	\begin{Detail}
		\SubBulletItem
		Tesis:
		\href{https://www.icloud.com/iclouddrive/0cfgD6maOU4-X1y5lNikwk4eQ#PEAG-Tesis_final}
		{Predicción \textit{in-silico} de los mecanismos de la glicólisis de polietilentereftalato con sales de zinc como catalizadores: un estudio en DFT ωB97X-D4/def2-TZVP}
		\SubSubBulletItem
		Asesor:
		Dra. Isabel del Carmen Sáenz Tavera
		\SubSubBulletItem
		Co-Asesor:
		Dr. Víctor Manuel Rosas García
		\SubSubBulletItem
		Enfoque:
		Química computacional, DFT, depolimerización, catálisis.

    \Gap

		\SubBulletItem
		\textbf{Examen de egreso (CENEVAL):}\\
		\textit{\textbf{Promedio:}} Sobresaliente\\
		\textit{\textbf{Identificación IES:}} 212655
		\SubSubBulletItem Sección transversal del lenguaje y comunicación: Sobresaliente;
		\SubSubBulletItem EGEL-QUIM: Sobresaliente.
	\end{Detail}


	
%%%%%%%%%%%%%%%%%%%%%%%%%
%% RESEARCH EXPERIENCE %%
%%%%%%%%%%%%%%%%%%%%%%%%%

\Section
{Experiencia en investigación}
{Experiencia en investigación}
{PDF:Experiencia en investigación}

\Entry
\href{http://www.fcq.uanl.mx/oferta-educativa/posgrado/}
{\textbf{División de Estudios de Posgrado, Facultad de Ciencias Químicas}},
UANL

\Gap
\BulletItem
Verano de investigación PROVERICYT
\hfill
\DatestampYMD{2017}{04}{1} --
\DatestampYMD{2017}{11}{1}
\begin{Detail}
	\SubBulletItem
	Proyecto:
	Síntesis de nanopartículas de \ch{FeS2} y \ch{NiO} por sonoquímica asistida con líquido iónico y evaluación de su desempeño como cátodo y ánodo en baterías de ión litio, respectivamente.
	\SubBulletItem
	Supervisor:
	Dra.~Salomé Maribel de la Parra Arciniega\\
	Colaboradores: Ariel Gonzalo Godoy Palominos; Mario Alberto Solano Esquer; Dr. Ignacio González Martínez; Dra. Rosa Martha Jiménez Barrera.
	\SubBulletItem
	Enfoque:
	Nanopartículas, sonoquímica, baterías, liquidos iónicos.
\end{Detail}

\Gap
\BulletItem
Estancia de investigación avanzada
\hfill
\DatestampYMD{2019}{08}{1} -- Presente
\begin{Detail}
	\SubBulletItem
	Proyecto:
	Predicción \textit{in-silico} de los mecanismos de la glicólisis de polietilentereftalato con sales de zinc como catalizadores: un estudio en DFT ωB97X-D4/def2-TZVP.
	\SubBulletItem
	Supervisor:
	Dra. Isabel del Carmen Sáenz Tavera\\
	Colaboradores: Dr. Víctor Manuel Rosas García.
	\SubBulletItem
	Enfoque:
	Química computacional, DFT, depolimerización, catálisis.
\end{Detail}


%%%%%%%%%%%%%%%%%%
% PUBLICATIONS
%%%%%%%%%%%%%%%%%%

	% \Section
	% {Publications}
	% {Publications}
	% {PDF:Publications}

	% \SubSection
	% {Journals}
	% {Journals}
	% {PDF:Journals}

	% % Declare a new group to limit the scope of \MaxNumberedItem to this subsection.
	% \begingroup
	% \renewcommand{\MaxNumberedItem}{[88]}

	% \BigGap
	% \NumberedItem{[10]}
	% \href{http://www.example.com/my-paper-doi-5}
	% {\underline{J.~Doe}, J.~Citizen, and A.~Yone,
	% ``On lasers and climate change,''
	% \textit{Journal of Science},
	% vol.~89,
	% no.~2,
	% pp.~4123--4133,
	% \DatestampYM{2008}{02}.}

	% \Gap
	% \NumberedItem{[1]}
	% \href{http://www.example.com/my-paper-doi-4}
	% {\underline{J.~Doe} and J.~Citizen,
	% ``Measuring the extent of climate change,''
	% \textit{Global Scientific Journal},
	% vol.~12,
	% no.~4,
	% pp.~330--352,
	% \DatestampYM{2006}{12}.}

	% \endgroup

	% \BigGap
	% \SubSection
	% {Conferences}
	% {Conferences}
	% {PDF:Conferences}

	% % Declare a new group to limit the scope of \MaxNumberedItem to this subsection.
	% \begingroup
	% \renewcommand{\MaxNumberedItem}{[8888]}

	% \BigGap
	% \NumberedItem{[1000]}
	% \href{http://www.example.com/my-paper-doi-3}
	% {\underline{J.~Doe}, J.~Citizen, and A.~Yone,
	% ``On lasers and climate change,''
	% in \textit{Proceedings of the Laser Symposium},
	% Las Vegas, Nevada, USA,
	% \DatestampYM{2007}{01}.}

	% \Gap
	% \NumberedItem{[100]}
	% \href{http://www.example.com/my-paper-doi-2}
	% {A.~Yone and \underline{J.~Doe},
	% ``Climate change and general relativity,''
	% in \textit{Proceedings of the International Astronomical Conference},
	% Sydney, Australia,
	% \DatestampYM{2006}{8}.}

	% \Gap
	% \NumberedItem{[10]}
	% \href{http://www.example.com/my-paper-doi-1}
	% {\underline{J.~Doe} and J.~Citizen,
	% ``Measuring the extent of climate change,''
	% in \textit{Proceedings of the International Climate Change Conference},
	% London, UK,
	% \DatestampYM{2005}{11}.}

	% \endgroup

%%%%%%%%%%%%%%%%%%%%%%%%%%%
%% Reconocimientos
%%%%%%%%%%%%%%%%%%%%%%%%%%%

\Section
{Diplomas}
{Diplomas}
{PDF:Diplomas}

%\BulletItem
%24va. Olimpiada de Química
%\hfill
%\DatestampYMD{2014}{11}{22}
%\begin{Detail}
%	\Item
%	Por la participación en el verano de investigación 24va. Olimpiada de Química, etapa estatal.
%\end{Detail}

% \Gap

\BulletItem
Verano de investigación, PROVERICYT
\hfill
\DatestampYMD{2017}{11}{1}
\begin{Detail}
	\Item
	Por la participación en el verano de investigación.
\end{Detail}

\Gap

\BulletItem
Ponencia en tema de seguridad en informática
\hfill
\DatestampYMD{2014}{24}{4}
\begin{Detail}
	\Item
	Por la participación con la ponencia en el tema de seguridad en informática en el XIV congreso Estudiantil inter-preparatorias en el área del lenguaje.
\end{Detail}

\Section
{Cursos y seminarios}
{Cursos y seminarios}
{PDF:CursosYseminarios}

\Entry
{\textbf{IV Congreso Internacional de Química e Ingeniería Verde}},
\newline
Facultad de Ciencias Químicas, UANL

\Gap
\BulletItem
Asistente
\hfill
\DatestampY{2017}

\Entry
{\textbf{V Congreso Internacional de Química e Ingeniería Verde}},
\newline
Facultad de Ciencias Químicas, UANL

\Gap
\BulletItem
Ponente 
\hfill
\DatestampY{2019}

\Entry
{\textbf{7th Orca User Meeting}},
\newline
Max-Planck-Institut für Kohlenforschung

\Gap
\BulletItem
Asistente
\hfill
\DatestampY{2021}

\Entry
{\textbf{Data Science in Real Life}},
\newline
Johns Hopkins University
\hfill
\DatestampY{2022}
\Section
{Certificaciones}
{Certificaciones}
{PDF:Certificaciones}

\Entry
    {\textbf{Basics of Database Design \& Development}},
    \newline
    Udemy
    \Gap
    \BulletItem
    ID de la credencial: UC-5bc6707c-ebc1-4787-b8cd-c1870ffee08f
    URL de la credencial: \url{https://www.udemy.com/certificate/UC-5bc6707c-ebc1-4787-b8cd-c1870ffee08f/}
    \hfill
    \DatestampY{2022}

\Entry
    {\textbf{V Congreso Internacional de Química e Ingeniería Verde}},
    \newline
    Facultad de Ciencias Químicas, UANL
\Gap
    \BulletItem
    Ponente
    \hfill
    \DatestampY{2019}

\Entry
    {\textbf{7th Orca User Meeting}},
    \newline
    Max-Planck-Institut für Kohlenforschung

    \Gap
    \BulletItem
    Asistente
    \hfill
    \DatestampY{2021}

\Entry
{\textbf{Data Science in Real Life}},
\newline
Johns Hopkins University
\hfill
\DatestampY{2022}

%%%%%%%%%%%%%%%%%%%%%%%%%%%%%%%%%%%%%%%%%%%%
%% Actividad profesional
%%%%%%%%%%%%%%%%%%%%%%%%%%%%%%%%%%%%%%%%%%%%

\Section
{Experiencia labroal}
{Experiencia labroal}
{PDF:ExperienciaLabroal}

% \Entry
% {\textbf{Sociedad Estudiantil: SELQI}},
% \newline
% Facultad de Ciencias Químicas, UANL

% \Gap
% \BulletItem
% Miembro
% \hfill
% \DatestampY{2019} -- \DatestampY{2020}

\Entry
{\textbf{Red de Fríos S.A. de C.V.}},
\newline
I. Comonfort, Bella Vista, 64410 Monterrey, N.L.

\Gap
\BulletItem
\textbf{\textit{Puesto:}} Jefe de aseguramiento de calidad
\hfill
\DatestampYYYYMM{2022-05} -- \textit{Actualmente}
\Gap
\begin{Detail}
\BulletItem \textbf{Actividades:}
    Las actividades cotidianas de mi departamento involucran: \compac elaboración y mantenimiento de información documentada, implementación y verificación de instrucciones de trabajo, implementación de programas (HACCP, VACCP, Food Fraud, Food Defense), generación y mantenimiento de formularios y sus versiones.
\Gap
    \BulletItem \textbf{Implementaciones al sistema de gestión de calidad:}
    Hasta la fecha me he encargado de llevar al día los registros generados a partir de procedimientos establecidos en el SGC. A su vez he implementado al SGC lo siguiente:
    \SubBulletItem
        \textbf{Control de versiones:} Mediante el uso de GIT, desarrollé un control de versiones para los formularios existentes e información documentada en general; 
    \SubBulletItem
        \textbf{digitalización de registros:} previamente los registros generados se mantenían exclusivamente en físico, con el propósito de prevenir perdida de información, digitalicé a partir de mi entrada todos los registros generados;
    \SubBulletItem
        \textbf{digitalización de registros:} previamente, la información documentada era difícil de consultar debido a que se encontraba en un formato inadecuado (*.docx) y presentaba referencias a artículos y a procedimientos obsoletos. Impulsé a que se cambiaran los procedimientos a Markdown para su eficaz consulta y a (Xe)LaTeX para la generación de la información documentada física. Desde entonces, la actualización de procedimientos y de folios se ha vuelto más sencilla;
    \SubBulletItem
        \textbf{bases de datos:} implementé una serie de bases de datos con el propósito de controlar más eficazmente la información obtenida de los registros;
    \SubBulletItem
        \textbf{análisis estadístico:} implementé el uso de pruebas estadísticas frecuentistas para verificar la variabilidad entre termómetros y entre mediciones a diferentes horas del día;
    \SubBulletItem
        \textbf{calibración de equipos de medición de temperatura:} verifico y calibro constantemente los termómetros con los que trabajamos (Termómetros IR, de vástago, de alcohol y termopares).    
\end{Detail}

	% Manual page break.


	%%%%%%%%%%%%%%%%%%%%%%%
	%% CAMPUS ACTIVITIES %%
	%%%%%%%%%%%%%%%%%%%%%%%

	%\input{sec.actcampus}

	%%%%%%%%%%%%%%%%%%%%%%%%%%%
	%% OTHER WORK EXPERIENCE %%
	%%%%%%%%%%%%%%%%%%%%%%%%%%%

	% \Section
	% {Other Work\newline
	% Experience}
	% {Other Work Experience}
	% {PDF:OtherWorkExperience}

	% \Entry
	% \href{http://www.example.com/my-company}
	% {\textbf{Alpha Engineering Firm}},
	% Oakland, Ohio, USA

	% \Gap
	% \BulletItem
	% Project Officer,
	% Department of Meteorological Sciences,
	% \hfill
	% \DatestampYMD{2007}{10}{15} --
	% \DatestampYMD{2008}{01}{15}
	% \newline
	% Research \& Development Division
	% \begin{Detail}
	% \SubBulletItem
	% Nullam venenatis egestas nisl eget elementum.
	% \SubBulletItem
	% Nulla finibus justo vel turpis efficitur, non lacinia orci maximus. Proin rhoncus, felis vel hendrerit lacinia, enim ipsum ultricies massa, sit amet interdum nisi massa sit amet justo.
	% \SubBulletItem
	% Etiam vitae eros mollis, consectetur quam quis, molestie massa.
	% \end{Detail}

	%%%%%%%%%%%%%%%
	%% LANGUAGES %%
	%%%%%%%%%%%%%%%

	\Section
{Idiomas}
{Idiomas}
{PDF:Languages}

\BulletItem
Español: Lengua nativa;

\Gap
\BulletItem
Inglés: Fluido (lectura, escritura), bueno (hablado).

\Gap
\BulletItem Francés: A2.


	%%%%%%%%%%%%
	%% SKILLS %%
	%%%%%%%%%%%%

	
\Section
{Habilidades}
{Habilidades}
{PDF:Skills}

\Entry
\subsection*{Software técnico}
{\TeX}, {\LaTeX}, {\XeLaTeX}, (Bueno);
MATLAB (Básico);
Mathematica (Básico);
Maple (Básico);
Python (Básico);
SAS (Básico);
Minitab;
JMP Pro;
GNUPlot;
Suite de Office;
Adobe Illustrator;
Adobe Photoshop;
Adobe Lightroom.

\Entry
\subsection*{Química computacional \& Modelado molecular}
Orca (Regular);
Gaussian (Básico);
XTB;
Jmol;
Avogadro;
MOLDEN;
JANPA;
NBO7;
Multiwfn.

\subsection*{Sistemas operativos}
MacOS (Bueno);
Windows (Bueno);
Linux (Bueno).

	%%%%%%%%%%%%%%%
	%% INTERESTS %%
	%%%%%%%%%%%%%%%

	%\Section
%{Intereses}
%{Intereses}
%{PDF:Interests}
%
%\Entry
%Fotografía digital y análoga,
%tipografía,
%jardinería,
%cocina y repostería.

	%%%%%%%%%%%%%%%%
	%% REFERENCES %%
	%%%%%%%%%%%%%%%%

	% \Section
	% {References}
	% {References}
	% {PDF:References}

	% \BulletItem
	% \textbf{Professor Jonathan Public}
	% \newline
	% Professor of Geology and Mechanical Engineering
	% \newline
	% First American University
	% \newline
	% 1000 First Avenue, Springfield, Massachusetts 22222, USA
	% \newline
	% \href{mailto:jonathanpublic@example.com}
	% {jonathanpublic@example.com}
	% \,\SubBulletSymbol\,
	% +1\,(555)\,222-2222

	% \BigGap
	% \BulletItem
	% \textbf{Dr Alice Bob Carol}
	% \newline
	% Director, Research \& Development
	% \newline
	% Alpha Engineering Firm
	% \newline
	% 20 North Street, Oakland, Ohio 33333, USA
	% \newline
	% \href{mailto:alicebobcarol@example.com}
	% {alicebobcarol@example.com}
	% \,\SubBulletSymbol\,
	% +1\,(555)\,333-3333

	% %%%%%%%%%%%%%%%%%%%%%%%%%%%%%%%%%%%
	% %% MULTILINGUAL UNICODE EXAMPLES %%
	% %%%%%%%%%%%%%%%%%%%%%%%%%%%%%%%%%%%

	% \Section
	% {Multilingual Unicode Examples}
	% {Multilingual Unicode Examples}
	% {PDF:MultilingualUnicodeExamples}

	% \BulletItem
	% Assortment of unicode characters from
	% \href{http://www.ltg.ed.ac.uk/~richard/unicode-sample.html}
	% {\url{http://www.ltg.ed.ac.uk/~richard/unicode-sample.html}}

	% \begin{Detail}
	% \Item
	% \textbf{Latin Extended-A}
	% Ā ā Ă ă Ą ą Ć ć Ĉ ĉ Ċ ċ Č č Ď ď Đ đ Ē ē Ĕ ĕ Ė ė Ę ę Ě ě Ĝ ĝ Ğ ğ Ġ ġ Ģ ģ Ĥ ĥ Ħ ħ Ĩ ĩ Ī ī Ĭ ĭ Į į İ ı IJ ij Ĵ ĵ
	% \textbf{Latin Extended-B}
	% ƀ Ɓ Ƃ ƃ Ƅ ƅ Ɔ Ƈ ƈ Ɖ Ɗ Ƌ ƌ ƍ Ǝ Ə Ɛ Ƒ ƒ Ɠ Ɣ ƕ Ɩ Ɨ Ƙ ƙ ƚ ƛ Ɯ Ɲ ƞ Ɵ Ơ ơ Ƣ ƣ Ƥ ƥ Ʀ Ƨ ƨ Ʃ ƪ ƫ Ƭ ƭ Ʈ Ư ư Ʊ Ʋ Ƴ ƴ Ƶ
	% \textbf{Latin Extended Additional}
	% Ḁ ḁ Ḃ ḃ Ḅ ḅ Ḇ ḇ Ḉ ḉ Ḋ ḋ Ḍ ḍ Ḏ ḏ Ḑ ḑ Ḓ ḓ Ḕ ḕ Ḗ ḗ Ḙ ḙ Ḛ ḛ Ḝ ḝ Ḟ ḟ Ḡ ḡ Ḣ ḣ Ḥ ḥ Ḧ ḧ Ḩ ḩ Ḫ ḫ Ḭ ḭ Ḯ ḯ Ḱ ḱ Ḳ ḳ Ḵ ḵ
	% \textbf{Greek}
	% ʹ ͵ ͺ ; ΄ ΅ Ά · Έ Ή Ί Ό Ύ Ώ ΐ Α Β Γ Δ Ε Ζ Η Θ Ι Κ Λ Μ Ν Ξ Ο Π Ρ Σ Τ Υ Φ Χ Ψ Ω Ϊ Ϋ ά έ ή ί ΰ α β γ δ ε ζ η θ
	% \textbf{Cyrillic}
	% Ё Ђ Ѓ Є Ѕ І Ї Ј Љ Њ Ћ Ќ Ў Џ А Б В Г Д Е Ж З И Й К Л М Н О П Р С Т У Ф Х Ц Ч Ш Щ Ъ Ы Ь Э Ю Я а б в г д е ж з
	% \textbf{Hebrew}
	% א ב ג ד ה ו ז ח ט י ך כ ל ם מ ן נ ס ע ף פ ץ צ ק ר ש ת װ ױ ײ ֝ ֞ ֟ ֠ ֡ ֣ ֤ ֥ ֦ ֧ ֨ ֩ ֪ ֫ ֬ ֭ ֮ ֯ ְ ֱ ֒ ֓ ֔
	% \textbf{Armenian}
	% {\UseSecondaryFont
	% Ա Բ Գ Դ Ե Զ Է Ը Թ Ժ Ի Լ Խ Ծ Կ Հ Ձ Ղ Ճ Մ Յ Ն Շ Ո Չ Պ Ջ Ռ Ս Վ Տ Ր Ց Ւ Փ Ք Օ Ֆ ՙ ՚ ՛ ՜ ՝ ՞ ՟ ա բ գ դ ե զ}
	% \textbf{Thai}
	% {\UseSecondaryFont
	% ก ข ฃ ค ฅ ฆ ง จ ฉ ช ซ ฌ ญ ฎ ฏ ฐ ฑ ฒ ณ ด ต ถ ท ธ น บ ป ผ ฝ พ ฟ ภ ม ย ร ฤ ล ฦ ว ศ ษ ส ห ฬ อ ฮ ฯ ะ ั า ำ ิ}
	% \end{Detail}

	% % Manual page break.
	% \newpage

	% %%%%%%%%%%%%%%%%%%%%%%%%%%%%%%%%%%%%%%%%
	% %% THIS IS A SECTION WITH USAGE NOTES %%
	% %%%%%%%%%%%%%%%%%%%%%%%%%%%%%%%%%%%%%%%%

	% % Declare a new group to limit the scope of \color to this section.
	% \begingroup
	% \color{red}

	% \Section
	% {This is a\newline
	% Section\newline
	% With\newline
	% Usage Notes}
	% {This is a Section With Usage Notes (For PDF Bookmark)}
	% {PDF:ThisIsASectionWithUsageNotes:ForPDFLink}

	% \SubSection
	% {This is a SubSection}
	% {This is a SubSection (For PDF Bookmark)}
	% {PDF:ThisIsASubSection:ForPDFLink}

	% \BigGap
	% \BulletItem
	% Use \CodeCommand{Section\{a\}\{b\}\{c\}} and
	% \CodeCommand{SubSection\{a\}\{b\}\{c\}}
	% to create sections and subsections, where
	% \Code{a} is the heading displayed on the page,
	% \Code{b} is the PDF bookmark heading, and
	% \Code{c} is the internal PDF link (must be unique).
	% Sections and subsections will appear in the PDF bookmarks.
	% Note the CamelCase command names.

	% \Gap
	% \BulletItem
	% Use
	% \CodeCommand{Entry},
	% \CodeCommand{BulletItem},
	% \CodeCommand{SubBulletItem},
	% \CodeCommand{Item},
	% \CodeCommand{SubItem},
	% \CodeCommand{NumberedItem},
	% etc.,
	% to create entries in the main body of the CV.

	% \Gap
	% \BulletItem
	% Enclose entry details between
	% \CodeCommand{begin\{Detail\}} and
	% \CodeCommand{end\{Detail\}}
	% so that they are typeset in a smaller font.
	% \begin{Detail}
	% \Item
	% This is an example of entry detail text enclosed in a \Code{Detail} environment.
	% \end{Detail}

	% \Gap
	% \BulletItem
	% Use \CodeCommand{Gap} and \CodeCommand{BigGap} to insert vertical spaces between entries to improve layout.

	% \BigGap
	% \SubSection
	% {This is Another SubSection}
	% {This is Another Subsection (For PDF Bookmark)}
	% {PDF:ThisIsAnotherSubSection:ForPDFLink}

	% \BigGap
	% \Entry
	% This is a plain \CodeCommand{Entry},
	% followed by an \CodeCommand{hfill} and a date range
	% \hfill
	% \DatestampYM{2015}{10} --
	% \DatestampYM{2015}{12}

	% \Gap
	% \BulletItem
	% This is a \CodeCommand{BulletItem}.
	% \Item
	% This is an \CodeCommand{Item}, which has no bullet.
	% Note the alignment with the \CodeCommand{BulletItem} above.

	% \Gap
	% \SubBulletItem
	% This is a \CodeCommand{SubBulletItem}.
	% \SubItem
	% This is a \CodeCommand{SubItem}, which has no bullet.
	% Note the alignment with the \CodeCommand{SubBulletItem} above.

	% \Gap
	% \NumberedItem{[42]}
	% This is a \CodeCommand{NumberedItem}.
	% Change the value of the macro \CodeCommand{MaxNumberedItem} to adjust the indentation width.

	% \BigGap
	% \SubSection
	% {Line, Paragraph, and Page Breaks}
	% {Line, Paragraph, and Page Breaks (For PDF Bookmark)}
	% {PDF:LineParagraphAndPageBreaks:ForPDFLink}

	% \BigGap
	% \BulletItem
	% To create a new line within the same paragraph (i.e., preserving the same paragraph indentation), use \CodeCommand{newline} instead of \CodeCommand{\textbackslash};
	% the latter will reset the paragraph indentation.

	% \Gap
	% \BulletItem
	% To create a new paragraph, use \CodeCommand{par} or simply leave an empty line.
	% Paragraph indentations (from
	% \CodeCommand{Entry},
	% \CodeCommand{BulletItem},
	% \CodeCommand{SubBulletItem},
	% \CodeCommand{Item},
	% \CodeCommand{SubItem},
	% \CodeCommand{NumberedItem},
	% etc.) do not carry across different paragraphs.

	% \Gap
	% \BulletItem
	% To create a new page, use \CodeCommand{newpage}.

	% \BigGap
	% \SubSection
	% {Dates}
	% {Dates (For PDF Bookmark)}
	% {PDF:Dates:ForPDFLink}

	% \BigGap
	% \BulletItem
	% Use the following macros to specify and display dates consistently:
	% \SubBulletItem
	% \CodeCommand{DatestampYMD\{yyyy\}\{MM\}\{dd\}}
	% (e.g., \CodeCommand{DatestampYMD\{2008\}\{01\}\{15\}})
	% \SubBulletItem
	% \CodeCommand{DatestampYM\{yyyy\}\{MM\}}
	% (e.g., \CodeCommand{DatestampYM\{2008\}\{01\}})
	% \SubBulletItem
	% \CodeCommand{DatestampY\{yyyy\}}
	% (e.g., \CodeCommand{DatestampY\{2008\}})

	% \Gap
	% \BulletItem\alert{text}
	% Change the date format option passed to the document class to adjust how dates are displayed throughout the document:
	% \SubBulletItem
	% \Code{MMMyyyy} (``Jan~2008'')
	% \SubBulletItem
	% \Code{ddMMMyyyy} (``15~Jan~2008'')
	% \SubBulletItem
	% \Code{MMMMyyyy} (``January~2008'')
	% \SubBulletItem
	% \Code{ddMMMMyyyy} (``15~January~2008'')
	% \SubBulletItem
	% \Code{yyyyMMdd} (``2008-01-15'')
	% \SubBulletItem
	% \Code{yyyyMM} (``2008-01'')
	% \SubBulletItem
	% \Code{yyyy} (``2008'')

	% \endgroup

\end{Body}

% %%%%%%%%%%%
% % CV NOTE %
% %%%%%%%%%%%

% \BigGap
% \UseNoteFont%
% \null\hfill%
% [\textit{\CVNote}]


\end{document}
